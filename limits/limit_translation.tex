\documentclass{article}
\usepackage[utf8]{inputenc}
\usepackage{amsmath}
\usepackage{amssymb}
\usepackage{amsfonts}
\usepackage{amstext}
\usepackage{amsthm}
\begin{document}

\section*{}
\textbf{Claim:} If the \(\lim\limits_{x \to c} f(x) = L\), then \(\lim\limits_{h \to 0} f(c + h) = L\)
\begin{proof}
    Let \(\epsilon > 0\) be given and choose \(\delta\) such that if \(0 < |x - c| < \delta\), then \(|f(x) - L| < \epsilon\). \\ \\
    Then, for any \(h\) satisfying \(0 < |h - 0| < \delta\), we have that \(0 < |(h + c) - c| < \delta\), which implies that \(|f(h + c) - L| < \epsilon\), as needed.
\end{proof}

\end{document}
